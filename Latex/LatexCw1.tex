\documentclass[12pt, letterpaper, titlepage]{article}
\usepackage[left=3.5cm, right=2.5cm, top=2.5cm, bottom=2.5cm]{geometry}
\usepackage[MeX]{polski}
\usepackage[utf8]{inputenc}
\usepackage{graphicx}
\usepackage{enumerate}
\usepackage{amsmath} %pakiet matematyczny
\usepackage{amssymb} %pakiet dodatkowych symboli
\usepackage{caption}
\captionsetup[table]{labelformat=empty}
\title{Pierwszy dokument LaTeX}
\author{Piotr Wielgolewski}
\date{Październik 2022}
\begin{document}
\maketitle
\section{Podstawowe Informacje}
Nazywam się Piotr Wielgolewski, Lorem ipsum dolor sit amet, consectetur adipiscing elit. Vivamus commodo quam sit amet scelerisque tincidunt. Aliquam at ante lorem. Vestibulum volutpat rhoncus quam eget condimentum. Proin leo elit, semper eu leo eu, molestie posuere augue. Nulla blandit faucibus efficitur. In euismod, purus vitae sollicitudin interdum, lectus quam commodo tortor, eget fringilla mauris dolor egestas lectus. Maecenas id facilisis ex. Donec hendrerit aliquam tincidunt. Cras efficitur velit vitae lorem pharetra, quis lacinia metus maximus. Donec sit amet leo iaculis, tristique urna ut, rutrum nulla. 
\subsection{Info}
Lorem ipsum dolor sit amet, consectetur adipiscing elit. Vivamus commodo quam sit amet scelerisque tincidunt. Aliquam at ante lorem. Vestibulum volutpat rhoncus quam eget condimentum. Proin leo elit, semper eu leo eu, molestie posuere augue. Nulla blandit faucibus efficitur. In euismod, purus vitae sollicitudin interdum, lectus quam commodo tortor, eget fringilla mauris dolor egestas lectus. Maecenas id facilisis ex. Donec hendrerit aliquam tincidunt. Cras efficitur velit vitae lorem pharetra, quis lacinia metus maximus. Donec sit amet leo iaculis, tristique urna ut, rutrum nulla. 
\begin{enumerate}
\item Aliquam ultrices nisi dui
\item Aliquam ultrices nisi dui
\end{enumerate}
\subsubsection{Lorem ipsum}
Suspendisse placerat massa finibus, egestas nisi eget, pretium dolor. Phasellus mollis quam eu augue finibus luctus. Duis auctor fermentum enim, nec ornare odio varius in. Aliquam erat volutpat. Morbi malesuada auctor eros at tincidunt. Quisque porttitor posuere mi, ut elementum neque vulputate ac. Sed tristique justo et ultrices scelerisque. Aliquam sodales bibendum arcu ut imperdiet. Pellentesque et odio vitae erat auctor aliquet. Pellentesque habitant morbi tristique senectus et netus et malesuada fames ac turpis egestas. Curabitur nec ultrices sem. Phasellus a molestie lectus. Suspendisse feugiat, enim nec fermentum venenatis, risus justo molestie mauris, vel fermentum lectus elit at lacus. Etiam ac nulla vel metus blandit dapibus. Nullam nec egestas enim, vitae lobortis lorem. 
\newpage

\section{Przepis na Bigos z kiszonej kapusty}

Składniki:
\begin{enumerate}
\item 500 g mięsa wieprzowego (np. karkówki)
\item 200 g kiełbasy
\item 1 cebula
\item 2 łyżki oleju roślinnego
\item 3 szklanki bulionu lub wody
\item 30 g suszonych borowików
\item 2 łyżki powideł śliwkowych lub kilka suszonych śliwek
\item 1 jabłko (np. reneta lub antonówka) - opcjonalnie
\item 1 kg kiszonej kapusty
\item 1 łyżka koncentratu pomidorowego
\item 1 łyżka mąki
\item 1 łyżka masła
\end{enumerate}
Pzyprawy:
\begin{enumerate}
\item 1 listek laurowy
\item 2 ziela angielskie
\item 1 łyżeczka kminku
\item 1 łyżka majeranku
\end{enumerate}
Sposób przygotowania:
\begin{enumerate}[•]
\item Mięso pokroić w kostkę. Cebulę pokroić w kosteczkę i zeszklić na oleju w dużym garnku. Dodać mięso i dokładnie je obsmażyć.
\item Wlać 2 szklanki gorącego bulionu lub wody z solą i pieprzem, zagotować. Następnie dodać połamane suszone grzyby, przykryć, zmniejszyć ogień i gotować przez ok.\textbf{45 minut}.
\item Dodać odciśniętą kiszoną kapustę oraz wlać szklankę wody, wymieszać. Przykryć i gotować przez ok.\textbf{15 minut}.
\item 	Kiełbasę obrać ze skóry, pokroić w kostkę i podsmażyć na patelni. Dodać do kapusty i gotować przez ok. \textbf{30 minut}. Pod koniec dodać koncentrat pomidorowy.
\item Mąkę podsmażyć na suchej patelni, gdy zacznie brązowieć dodać łyżkę masła i mieszać aż masło się rozpuści.
\item  Trzymając patelnię na ogniu dodać stopniowo kilka łyżek kapusty cały czas mieszając. Przełożyć zawartość patelni z powrotem do garnka, wymieszać i zagotować.
\end{enumerate}

\section{Przykładowy system decyzji (U, A, d)}

\begin{table}[h]
\centering\caption{Przykładowy system decyzji (U, A, d), modelujący problem diagnozy medycznej, której efektem jest decyzja o wykonani lub nie wykonaniu operacji wycięcia wyrostka robaczkowego U = \textit{{\{$u_1$, $u_2$, ..., $u_{10}$}\}, A = {\{$a_1$, $a_2$}\} ,  d $\in$ D = {\{TAK, NIE}\}}\newline}
\begin{tabular}{c| c c c}
\hline
\hline
Pacjent & Ból brzucha & Temperatura ciała & Operacja\\
\hline
u1 & Mocny & Wysoka & Tak\\

u2 & Średni & Wysoka & Tak\\

u3 & Mocny & Średnia & Tak\\

u4 & Mocny & Niska & Tak\\

u5 & Średni & Średnia & Tak\\

u6 & Średni & Średnia & Nie\\

u7 & Mały & Wysoka & Nie\\

u8 & mały & Niska & Nie\\

u9 & Mocny & Niska & Nie\\

u10 & Mały & Średnia & Nie\\

\hline
\hline
\end{tabular}
\end{table}

\section{Bramki logiczne}

\begin{table}[h]
\centering\caption{AND}
\begin{tabular}{c c| c}

A & B & Q\\
\hline
0 & 0 & 0\\
0 & 1 & 0\\
1 & 0 & 0\\
1 & 1 & 1\\

\end{tabular}
\end{table}

\begin{table}[h]
\centering\caption{OR}
\begin{tabular}{c c| c}

A & B & Q\\
\hline
0 & 0 & 0\\
0 & 1 & 1\\
1 & 0 & 1\\
1 & 1 & 1\\

\end{tabular}
\end{table}
\begin{table}[h]
\centering\caption{NAND}
\begin{tabular}{c c| c}

A & B & Q\\
\hline
0 & 0 & 1\\
0 & 1 & 1\\
1 & 0 & 1\\
1 & 1 & 0\\

\end{tabular}
\end{table}
\begin{table}[h]
\centering\caption{NOR}
\begin{tabular}{c c| c}

A & B & Q\\
\hline
0 & 0 & 1\\
0 & 1 & 0\\
1 & 0 & 0\\
1 & 1 & 0\\

\end{tabular}
\end{table}
\begin{table}[h]
\centering\caption{NOT}
\begin{tabular}{c| c}

A & Q\\
\hline
0 & 1\\
1 & 0\\


\end{tabular}
\end{table}
\begin{table}[h]
\centering\caption{XOR}
\begin{tabular}{c c| c}

A & B & Q\\
\hline
0 & 0 & 0\\
0 & 1 & 1\\
1 & 0 & 1\\
1 & 1 & 0\\

\end{tabular}
\end{table}
\end{document}